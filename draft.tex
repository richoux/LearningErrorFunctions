\documentclass[a4paper, 12pt]{article}

\usepackage[english]{babel}
\usepackage[T1]{fontenc} 
\usepackage[utf8]{inputenc}
\usepackage{fullpage}
\usepackage{url}
\usepackage{xspace}
\usepackage{hyperref}
\usepackage{graphicx} 
\usepackage{eurosym}
\usepackage{amsmath}
\usepackage{fourier}
\usepackage[normalem]{ulem}

\hypersetup{
  colorlinks   = true, %Colours links instead of ugly boxes
  urlcolor     = blue, %Colour for external hyperlinks
  linkcolor    = blue, %Colour of internal links
  citecolor   = blue %Colour of citations
}

\title{Draft v1 - Learning Cost Functions}
%\author{}
\date{\today}
 
\begin{document}
\maketitle

\includegraphics[width=\linewidth,angle=180]{20190529_whiteboard}

\section{Main ideas}
\begin{enumerate}
\item The user must first provide for each constraint type a function $b(\vec{x}) = \left\{
  \begin{array}{rl}
    0 & \text{if } x \text{ is a solution}\\
    1 & y\text{otherwise}
  \end{array} \right.$
$b$ is called ``concept'' in some papers.
\item The goal is to learn $f(\vec x)$ such that the cost function is defined by $b(\vec x) . f(\vec x)$.

\item Find a metric $|.|_l$ in the constraint configuration space. (\danger Space liked to the constraint, since depending one the variables and their domains.)
\item Project the current configuration $\vec{x}$ on the closest solution $\vec{s}$, giving $f(\vec x) = |\vec{s}-\vec{x}|_l$. Need to search for this closest $\vec{s}$ (n-dimensional Voronoï cells?).
\item Fourier transform of the cost function $f'$ of a well-known constraint like all-diff or equality to compare 'hand-designed' harmonics and learned ones.
\item Possible metrics:
  \begin{itemize}
  \item Manhattan: $\sum\limits_i |s[i] - x[i]|$
  \item Hamming: number of variables to change to get a solution (ideal for local search)
  \item Number of swaps of values to get a solution (ideal for permutation constraints).
  \item Manhattan/Hamming mix: Hamming first, then Manhattan as a tie-breaker.
  \end{itemize}
\item Brute-force solution search on small spaces: from a configuration $\vec{x}$, try all possible combinations and test if it is a solution. Early-stop: once we have found a solution, no need to continue the search.
\item The idea is to characterize the cost functions over small instances (few variables/small domains) to make it scale over larger instances.
\item Multivariate  interpolation to find  a scalable formula  for $f$
  (\sout{chebpol} splinter).
\item Use of Genetic Algorithms/Programming, other Evolutionary Computation, Reinforcement learning, Supervised learning?
\end{enumerate}

\section{Sidekick ideas}
\begin{itemize}
\item[1 bis] The user provides some examples $\vec{x}$ 1. near to be a solution, 2. not a solution but not ridiculous and 3. being a really bad configuration.
\item[3 bis] Could be simplified by a linear transform contracting the constraint configuration space on the manyfold of solutions (so det = 0). Only possible if the manyfold is linear?
\item[5 bis] Random draw of solutions to estimate a projection. Need to know and save solutions somewhere (or to be able to find them quickly).
\item [6 bis] See also Kantorovich functional distance from~\cite{StochCP}. See also~\cite{metrics}.
\item [8 bis] Generate training examples and do reinforcement learning/supervised learning (regression problem)? keeping in mind we look for a function based over the sum of harmonics.
\end{itemize}

\section{Processing data}
\begin{itemize}
\item Normalize samples or apply  a $log_2(x+2)-1$ scaling function to
  diminish potential gaps, in particular  when we have solutions (cost
  = 0).
\item Idea of backward JLL or ``kind of kernel trick'':
  \begin{enumerate}
  \item The user furnishes constraint concepts.
  \item We apply JLL to reduce the space size.
  \item We learn our cost function on this space.
  \item  We apply  some  ``backward  JLL'' to  this  cost function  to
    preserve its properties on the original space.
  \end{enumerate}
\end{itemize}

\section{Formalization}
Define WCSP and CFN such that:
\begin{itemize}
\item  WCSP is  considering soft  constraints, where  a constraint  is
  unsatisfyed iif its associated cost function outputs a value below a
  given threshold $k$ (can be infinite).
\item  CFN  is  considering  hard constraints  only.  Like  constraint
  networks helping solvers to find  solutions by giving a structure of
  the problem,  CFN gives,  in addition of  the constraint  network, a
  structure on  configurations to help  the solver to determine  if an
  unsatisfying configuration is near to be a solution or not.
\item Expressiveness: CFN-sat $<$ WCSP $<$ CFN-opt
\item Reductions/transformation: WCSP $\Rightarrow$ CFN-opt
\end{itemize}

\section{To explore}
\begin{itemize}
\item Can the Johnson-Lindenstrauss Lemma be useful here?
\item Dichotomy (!?)
\item Constraint classification regarding their cost function.
\item Compute the cost function of the main global constraints.
\item Study the robustness of the cost function regarding the ratio $\frac{|Domain|}{\# vars}$. Does normalization change something?
\item Compositional pattern-producing network (CPPN~\cite{CPPN}) instead of the sum of harmonics.
\end{itemize}

\section{Questions / issues}
\begin{itemize}
\item Discrete case of trigonometric functions.
\item How to scale over more variables, like $\text{all-diff}(x, y, z)
  \rightarrow \text{all-diff}(x_1, \ldots, x_n)$
\item How to display an interpoled function on chebpol?
\item Metrics  over the function  space, to  see if two  functions are
  similar or not:
  \begin{itemize}
  \item interpoled function over the  full search space vs. interpoled
    function over samples
  \item interpoled function vs. hand-designed function.
  \end{itemize}
\end{itemize}


\bibliographystyle{alpha}
\bibliography{draft}

\end{document}
